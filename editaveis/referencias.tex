
\chapter[Referências]{Referências}

Gestão de projeto

AErtebjerg, G., Andersen, J.H. e Hansen, O.S. (eds.) (2003) Nutrients and Eutrophication in Danish Marine Waters. A Challenge for Science and Management. National Environmental Research Institute. 126 pp.

Pesquisa dos componentes eletrônicos, subgrupo comunicação e análise de sensores

Projeto de um barco autonomo com arduino utilizando GPS:

http://minddump.us/2016/05/autopilot-rc-boat/
https://www.youtube.com/watch?v=SDhbZppc-JI

Como configurar o módulo Xbee com o Arduino:

http://blog.filipeflop.com/wireless/tutorial-wireless-arduino-xbee-shield.html

Tutorial básico Xbee e Arduino:

http://labdegaragem.com/profiles/blogs/tutorial-como-utilizar-o-xbee-com-arduino

Barco Autonomo feito com arduino:

http://www.instructables.com/id/Boat-Autopilot/

Tutorial de como usar um sensor de temperatura de baixo dagua com Arduino:

http://blog.filipeflop.com/sensores/sensor-de-temperatura-ds18b20-arduino.html

Link para comprar o sensor de temperatura pra agua:

http://www.baudaeletronica.com.br/sensor-de-temperatura-a-prova-de-agua-ds18b20-dfrobot.html?gclid=CjwKEAjw8OLGBRCklJalqKHzjQ0SJACP4BHrrggEQUZCaAaKJ9Cego6gq3RlTYGqVNv4HWNEo31cDRoC9-zw\_wcB

Controle de Motores RC configurando com um controle externo:

http://www.instructables.com/id/RC-Control-and-Arduino-A-Complete-Works/


Controle de motores e alimentação

SCHOEPING, Djonatan Guilherme Erbs. Projeto preliminar de sistema propulsivo de uma embarcação de apoio offshore do tipo Platform supply vessel. Universidade Federal de Santa Catarica, 2014.

BOAS, Fábio Vilas. Desenvolvimento de uma ferramenta de CAD aplicada ao projeto de hélices para veículos aquático não tripulaveis. São Paulo, 2006.

André L. Fuerback, Bruno S. Dupczak, Cesar A. Arbugeri, Paulo R. C. Villa, Walbermark M. dos Santos, Aline T. de Souza, Denizar C. Martins, Marcelo L. Heldwein, Samir A. Mussa, Arnaldo J. Perin. Sistema elétrico de propulsão para barcos de pequeno porte.

Yunus A. Çengel.John M. Cimbala.Mecânica dos Fluidos - Fundamentos e aplicações. São Paulo: McGraw-Hill, 2007.

Mercado Livre. Disponível em: <http://produto.mercadolivre.com.br/MLB-754558734-12v-dc-cpu-cooling-carro-brushless-agua-bomba-de-oleo-prov-\_JM> . Acesso em: 29 de Março de 2017.

ASSOCIAÇÃO BRASILEIRA DE NORMAS TÉCNICAS. NBR 9898/87: Preservação e Técnicas de Amostragem de Efluentes Líquidos e Corpos Receptores.  

Alterima. Disponível em: <http://www.alterima.com.br/index.asp?InCdSecao=38> . Acesso em: 29 de Março de 2017.

MACINTYRE, A, J. Bombas de instalações de bombeamento. 2 Edição. Ed. LTC, Rio de Janeiro, 1997.


Estrutura

ABAL - Associação Brasileira do Alumínio, Fundamentos e aplicações do alumínio. São Paulo, 2012.
 
CALLISTER, W. D. Ciência e Engenharia de Materiais: Uma Introdução. John Wiley \& Sons, Inc., 2002.
 
CARUSO. Tecnologia Mecânica. CEFET-SP, 2001. Disponível em: http://www.joinville.ifsc.edu.br/~paulosergio/Ciencia\_dos\_Materiais/Classifica\%C3\%A7\%C3\%A3o\%20dos\%20a\%C3\%A7os.pdf. Acessado em: 29 de março de 2017.

COSTA, Fernando Mustafá. Construção, reparo, conservação, manutenção e navegação em embarcações. Universidade de São Paulo – USP, 2012. Disponível em: http://www.sbrt.ibict.br/dossie-tecnico/downloadsDT/NTcxMA==. Acessado em: 28 de março de 2017.

NASSEH, Jorge. Barcos, Métodos avançados de construção em composites. Rio de Janeiro, 2007.

NASSEH, Jorge. Técnica e prática de laminação em composites. Rio de Janeiro, 2008.
 
NETO, Flamínio Levy; PARDINI, Luiz Claudio. Compósitos estruturais: ciência e tecnologia. 2006

FRANCESCHI, Alessandro. Elementos de máquinas / Alessandro de Franceschi, Miguel Guilherme Antonello. – Santa Maria, RS : Universidade Federal de Santa Maria, Colégio Técnico Industrial de Santa Maria : Rede e-Tec Brasil, 2014.

NORTON, Robert L. Projeto de máquinas: uma abordagem integrada / Robert L. Norton; tradução João Batista de Aguiar, ... [et al.]. - 2 ed. – Porto Alegre : Bookman, 2004.

SHIGLEY, Joseph E. Projeto de Engenharia Mecânica / Joseph E. Shigley, Charles R. Mischke, Richard G. Burynas: Tradução de João Batista de Aguiar, José Manoel de Aguiar. – 7 ed. – Porto Alegre: Bookman, 2005.